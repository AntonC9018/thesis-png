\documentclass[a4paper,12pt]{report}
\usepackage{config}

% Description
\newcommand{\authorName}{CURMANSCHII Anton}
\newcommand{\thesisTitle}{PNG}
\newcommand{\uniGroupName}{IA1901}
\newcommand{\thesisType}{master}
\newcommand{\programulDeStudii}{master}
\newcommand{\identificatorulCursului}{0613.5 Informatică aplicată}

% Hardcoded for now
\newcommand{\anexeCount}{20}
\newcommand{\conferencesList}{I don't know what to write here}

\begin{document}

% Prevent spills over margin
\sloppy

\begin{titlepage}
  \vspace*{\fill}
  \begin{center}
      \vspace*{1cm}

      \large
      \uppercase{\textbf{UNIVERSITATEA DE STAT DIN MOLDOVA\\}}

      \normalsize
      \uppercase{\textbf{FACULTATEA DE MATEMATICĂ și INFORMATICĂ}}
      \vspace{0.1cm}

      \normalsize
      \uppercase{\textbf{DEPARTAMENTUL INFORMATICĂ}}
      \vspace{3.0cm}

      \large
      \textbf{\expandafter{\authorName}}
      \vspace{1.5cm}

      \Large
      \textbf{\uppercase\expandafter{\thesisTitle}}
      \vspace{0.75cm}

      \large
      \textbf{\uppercase\expandafter{\identificatorulCursului}}
      \vspace{0.5cm}

      \normalsize
      \textbf{Teză de \programulDeStudii{}}
  
    \end{center}
  \vfill

  \normalsize
  
  \begin{flushleft}
    \begin{tabular}{p{4.3cm} p{4cm} p{8cm}}
      Director de departament: & \signatureField{4cm} & CAPCELEA Titu, \\
                               &                      & doctor în științe fizico-matematice, \\
                               &                      & conferențiar universitar \\
      Conducător științific:   & \signatureField{4cm} & \uppercase\expandafter{\conducatorNume\relax} \conducatorPrenume{}, \\
                               &                      & doctor, conferențiar universitar \\
      Autor:                   & \signatureField{4cm} & \authorName{}, \\
                               &                      & masterand din grupa \uniGroupName{} \\
    \end{tabular}
      
  \end{flushleft}

  \vspace{1cm}

  \begin{center}
      \textbf{\location{} -- \year{}}
  \end{center}

\end{titlepage}


\clearpage
\tableofcontents

\clearpage
\unnumberedChapter{Lista Abrevierelor}
\begin{acronym}
  \acro{CI/CD}{Continuous Integration / Continuous Delivery}
  \acro{GUI}{Graphical User Interface}
  \acro{PNG}{Portable Network Graphics}
  \acro{I/O}{Input/Output}
  \acro{MVP}{Produsul Minim Viabil}

\end{acronym}


\clearpage
\unnumberedChapter{Adnotare}

\textbf{la teza de \thesisType ``\thesisTitle'', a studentului \authorName{}, grupa \uniGroupName{}, programul de studii \programulDeStudii.}

\textbf{Structura tezei.}
Teza constă din: Introducere, \total{num_chapters} capitole, Concluzii generale și recomandări, Bibliografie \bibliographyEntryCount{} titluri.
Textul de bază cuprinde \usefulPageCount{} de pagini și \anexeCount{} de anexe.

\textbf{Cuvinte-cheie:}

\textbf{Actualitatea.}

\textbf{Scopul și obiectivele cercetării}.

\textbf{Rezultatele preconizate și obținute} rezumă în: (1) (2) (3)

\textbf{Problemele importante rezolvate} sunt:

\textbf{Valoarea aplicativă.}

Rezultatele obținute au fost raportate la Conferința (-ele) \textbf{\conferencesList}.

% I think this one's required to be capitalized.
\unnumberedChapter{INTRODUCERE}

\markpage{usefulStuffBegin}

\textbf{Actualitatea și importanța temei.}

\textbf{Scopul și obiectivele.}

\textbf{Suportul metodologic și tehnologic.}

\textbf{Noutatea stiințifică/originalitatea.}

\textbf{Valoarea aplicativă.}

\textbf{Sumarul tezei.}

Primul capitol, \nameref{intro_chapter_title}, aduce informații generale.
Al doilea capitol, \nameref{architecture_chapter_title}, concretizează cerințele.
Al treilea capitol, \nameref{implementation_chapter_title}, urmează implementarea.


\chapter{Capitol introducere}\label{intro_chapter_title}

\chapterConclusionSection{intro_chapter_title}


\chapter{Arhitectura Aplicației}\label{architecture_chapter_title}

\chapterConclusionSection{architecture_chapter_title}


\chapter{Implementarea Sistemului}\label{implementation_chapter_title}

În continuare, se va prezenta implementarea unui decoder PNG care va fi folosit
pentru a vizualiza formatul PNG printr-o aplicație \ac{GUI}. 
Decoderul va fi implementat în limbajul de programare Zig\cite{zig} după specificația \ac{PNG}, versiunea 1.2\cite{png_spec}.
Interfața grafică va fi implementată folosind Raylib\cite{raylib}, tot în Zig.

Deoarece autorul nu a cunoscut atât de bine limbajul Zig anterior,
și deoarece a dorit să facă o abstracție pentru citire a stream-urilor de date,
a fost implementată o librărie mică pentru aceasta.
Autorul a lucrat anterior cu librăria \texttt{System.IO.Pipelines}\cite{system_io_pipelines} din C\#,
și a observat că codul scris folosind această librărie are următoarele avantaje 
asupra codului obișnuit:
\begin{itemize}
  \item 
    Realizarea unei implementări corecte care se ocupă de toate cazurile-limită 
    este ușurată datorită faptului că partea majoră a complexității este gestionată de librărie.
    Ca exemplu, librăria se ocupă cu alocarea și ștergerea buferilor,
    precum și înfășurarea datelor pe mai multe segmente consecutive.

  \item 
    Duce la un cod mai flexibil, unde implementarea de obicei la sfârșit ajunge 
    la un automat finit.
    Aceasta implică că codul care prelucrează datele poate
    fi oprit și pe urmă rerulat dintr-o careva stare sălvată anterior.

  \item
    Duce la un cod mai eficient, deoarece se soluționează cazurile unde să se aibă nevoie de
    a copia datele într-un bufer temporar, sau a stoca mai multe sau mai puține date deoadată
    decât este optim.

  \item
    Duce la centralizarea logicii de citire în formă de un ciclu centralizat de citire a datelor.
    Acesta duce la reducerea duplicării a codului de citire.

  \item
    Permite separearea completă a modulelor de citire și de scriere.
    Datele pot fi preluate într-un fir separat de cel care le procesează,
    ceea ce elimină timpul pierdut la \ac{I/O} în timpul procesării.
\end{itemize}
    
Ca prototip, autorul a hotărât să realizeze o implementare a unei librării similare în Zig,
însă fără ultimul punct de separare a modulelor de citire și de scriere,
deoarece acesta ar adăuga destul de multă complexitate din cauza că s-ar trebui
să se sincronizeze acele fire, deoarece datele ar fi scrise în memorie partajată.
Partea această de cod poate fi realizată mai târziu și nu este crucială pentru un \ac{MVP}.



\chapterConclusionSection{implementation_chater_title}


\unnumberedChapter{Concluziile Finale și Recomandările}


\newpage
\markpage{usefulStuffEnd}


% Bibliography
\bibliographystyle{plain}
\bibliography{bibliography}
\addcontentsline{toc}{chapter}{\bibname}

% Appendices
\appendix

% Number with arabic numbers instead of Roman
\renewcommand{\thechapter}{\arabic{chapter}}
% Prepend Anexa to section names, center them
\titleformat{\section}[block]{\normalfont\normalsize\bfseries\filcenter}{Anexa \thesection~}{0pt}{}

% Every section on new page
% \newcommand{\sectionbreak}{\clearpage}

% Since we've got just a single chapter in the appedices,
% but which is also the name of the Appendix chapter, it should be omitted.
% Makes little sense, but ok I guess.
\setcounter{chapter}{1}

\unnumberedChapter{Anexe}
% insert appendices here

\end{document}