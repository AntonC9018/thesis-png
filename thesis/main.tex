\documentclass[a4paper,12pt]{report}
\usepackage{config}

% Description
\newcommand{\authorName}{CURMANSCHII Anton}
\newcommand{\thesisTitle}{PNG}
\newcommand{\uniGroupName}{IA1901}
\newcommand{\thesisType}{master}
\newcommand{\programulDeStudii}{master}
\newcommand{\identificatorulCursului}{0613.5 Informatică aplicată}

% Hardcoded for now
\newcommand{\anexeCount}{20}
\newcommand{\conferencesList}{I don't know what to write here}

\begin{document}

% Prevent spills over margin
\sloppy

\begin{titlepage}
  \vspace*{\fill}
  \begin{center}
      \vspace*{1cm}

      \large
      \uppercase{\textbf{UNIVERSITATEA DE STAT DIN MOLDOVA\\}}

      \normalsize
      \uppercase{\textbf{FACULTATEA DE MATEMATICĂ și INFORMATICĂ}}
      \vspace{0.1cm}

      \normalsize
      \uppercase{\textbf{DEPARTAMENTUL INFORMATICĂ}}
      \vspace{3.0cm}

      \large
      \textbf{\expandafter{\authorName}}
      \vspace{1.5cm}

      \Large
      \textbf{\uppercase\expandafter{\thesisTitle}}
      \vspace{0.75cm}

      \large
      \textbf{\uppercase\expandafter{\identificatorulCursului}}
      \vspace{0.5cm}

      \normalsize
      \textbf{Teză de \programulDeStudii{}}
  
    \end{center}
  \vfill

  \normalsize
  
  \begin{flushleft}
    \begin{tabular}{p{4.3cm} p{4cm} p{8cm}}
      Director de departament: & \signatureField{4cm} & CAPCELEA Titu, \\
                               &                      & doctor în științe fizico-matematice, \\
                               &                      & conferențiar universitar \\
      Conducător științific:   & \signatureField{4cm} & \uppercase\expandafter{\conducatorNume\relax} \conducatorPrenume{}, \\
                               &                      & doctor, conferențiar universitar \\
      Autor:                   & \signatureField{4cm} & \authorName{}, \\
                               &                      & masterand din grupa \uniGroupName{} \\
    \end{tabular}
      
  \end{flushleft}

  \vspace{1cm}

  \begin{center}
      \textbf{\location{} -- \year{}}
  \end{center}

\end{titlepage}


\clearpage
\tableofcontents

\clearpage
\unnumberedChapter{Lista Abrevierelor}
\begin{acronym}
  \acro{CI/CD}{Continuous Integration / Continuous Delivery}
  \acro{GUI}{Graphical User Interface}
  \acro{PNG}{Portable Network Graphics}
  \acro{I/O}{Input/Output}
  \acro{MVP}{Produs Minim Viabil}

\end{acronym}


\clearpage
\unnumberedChapter{Adnotare}

\textbf{la teza de \thesisType ``\thesisTitle'', a studentului \authorName{}, grupa \uniGroupName{}, programul de studii \programulDeStudii.}

\textbf{Structura tezei.}
Teza constă din: Introducere, \total{num_chapters} capitole, Concluzii generale și recomandări, Bibliografie \bibliographyEntryCount{} titluri.
Textul de bază cuprinde \usefulPageCount{} de pagini și \anexeCount{} de anexe.

\textbf{Cuvinte-cheie:} PNG

\textbf{Actualitatea.}
PNG este un format de imagini folosit pe scară largă pe Internet,
în dezoltarea aplicațiilor grafice, în jocuri, și în multe alte domenii.
Înțelegerea structurii formatului poate fi utilă pentru a-l putea extinde, 
a-l utiliza într-un mod mai eficient și acesta poate servi drept referință pentru 
crearea unor formate noi.

\textbf{Scopul și obiectivele cercetării}.



\textbf{Rezultatele preconizate și obținute} rezumă în: (1) (2) (3)

\textbf{Problemele importante rezolvate} sunt:

\textbf{Valoarea aplicativă.}

Rezultatele obținute au fost raportate la Conferința (-ele) \textbf{\conferencesList}.

% I think this one's required to be capitalized.
\unnumberedChapter{INTRODUCERE}

\markpage{usefulStuffBegin}

\textbf{Actualitatea și importanța temei.}

\textbf{Scopul și obiectivele.}

\textbf{Suportul metodologic și tehnologic.}

\textbf{Noutatea stiințifică/originalitatea.}

\textbf{Valoarea aplicativă.}

\textbf{Sumarul tezei.}

Primul capitol, \nameref{intro_chapter_title}, aduce informații generale.
Al doilea capitol, \nameref{architecture_chapter_title}, concretizează cerințele.
Al treilea capitol, \nameref{implementation_chapter_title}, urmează implementarea.


\chapter{Capitol introducere}\label{intro_chapter_title}

\chapterConclusionSection{intro_chapter_title}


\chapter{Arhitectura Aplicației}\label{architecture_chapter_title}

\chapterConclusionSection{architecture_chapter_title}


\chapter{Implementarea Sistemului}\label{implementation_chapter_title}

În continuare, se va prezenta implementarea unui decoder PNG care va fi folosit
pentru a vizualiza formatul PNG printr-o aplicație \ac{GUI}. 
Decoderul va fi implementat în limbajul de programare Zig\cite{zig} după specificația \ac{PNG}, versiunea 1.2\cite{png_spec}.
Interfața grafică va fi implementată folosind Raylib\cite{raylib}, tot în Zig.

Deoarece autorul nu a cunoscut atât de bine limbajul Zig anterior,
și deoarece a dorit să facă o abstracție pentru citire a stream-urilor de date,
a fost implementată o librărie mică pentru aceasta.
Autorul a lucrat anterior cu librăria \texttt{System.IO.Pipelines}\cite{system_io_pipelines} din C\#,
și a observat că codul scris folosind această librărie are următoarele avantaje 
asupra codului obișnuit:
\begin{itemize}
  \item 
    Realizarea unei implementări corecte care se ocupă de toate cazurile-limită 
    este ușurată datorită faptului că partea majoră a complexității este gestionată de librărie.
    Ca exemplu, librăria se ocupă cu alocarea și ștergerea buferilor,
    precum și înfășurarea datelor pe mai multe segmente consecutive.

  \item 
    Duce la un cod mai flexibil, unde implementarea de obicei la sfârșit ajunge 
    la un automat finit.
    Aceasta implică că codul care prelucrează datele poate
    fi oprit și pe urmă rerulat dintr-o careva stare sălvată anterior.

  \item
    Duce la un cod mai eficient, deoarece se soluționează cazurile unde să se aibă nevoie de
    a copia datele într-un bufer temporar, sau a stoca mai multe sau mai puține date deoadată
    decât este optim.

  \item
    Duce la centralizarea logicii de citire în formă de un ciclu centralizat de citire a datelor.
    Acesta duce la reducerea duplicării codului de citire.

  \item
    Permite separearea completă a modulelor de citire și de scriere.
    Datele pot fi preluate într-un fir separat de cel care le procesează,
    ceea ce elimină timpul pierdut la \ac{I/O} în timpul procesării.
\end{itemize}
    
Ca prototip, autorul a hotărât să realizeze o implementare a unei librării similare în Zig,
însă fără ultimul punct de separare a modulelor de citire și de scriere,
deoarece acesta ar adăuga destul de multă complexitate din cauza că s-ar trebui
să se sincronizeze acele fire, deoarece datele ar fi scrise în memorie partajată.
Partea aceasta de cod poate fi realizată mai târziu și nu este crucială pentru un \ac{MVP}.


\section{Ideea aplicației la nivel înalt}

Cu toate că autorul a hotărât să întreprindă o abordare explorativă față de proiect,
adică să programeze aplicația având doar o imagine vagă a rezultatului dorit,
totuși ar fi dorit să fie evidențiată vederea generală a autorului asupra aplicației.

Având în vedere faptul că scopul aplicației este să poată explora structura fișierilor PNG,
funcționalitatea aplicației ar trebui să fie într-un fel impusă de structura acestora.
Se are în vedere faptul că modul de folosire și de interpretare a interfeței grafice 
ar trebui să fie inspirat de structura fișierelor PNG.

Fișierele PNG sunt constituite de așa numite chunk-uri,
unde fiecare chunk are o anumită destinație funcțională.
De exemplu, chunk-ul critic \texttt{IHDR} conține niște informații generale despre imagine,
ca lățimea și înălțimea, numărul de biți per pixel, ș.a.
Ideea autorului este ca interfața să evidențieze fiecare așa chunk,
oferind utilizatorului diferite informații despre acesta,
și permitându-i a modifica valorile, dacă lucrul acesta n-ar strica integritatea fișierului.
Informațiile ar fi prezentate per chunk, atunci când utilizatorul îl alege,
iar fiecare chunk trebuie să fie evidențiat, pentru a-l putea alege și vizualiza.

Deoarece este importantă legarea chunk-urilor la octeți din fișier, că se realizează explorarea
\textit{formatului} PNG, dar nu pur și simplu lucrul cu imagini folosind formatul PNG, s-ar dori să se aibă
un fel de editor hexazecimal, unde s-ar fi afișate valorile la fiecare octet,
precum și informația vizuală pentru a putea distinge chunk-urile.
Ar fi bine ca acesta să aibă funcția care să permită ca utilizatorul să poată micșora anumite chunk-uri,
apăsând o iconiță pe ele, deoarece mai multe chunk-uri conțin date de pixeli,
și nu sunt direct utile pentru înțelegerea formatului.

Alegerea unui chunk ar trebui să afișeze informații despre acesta,
ca numele, descrierea, lungimea ș.a.
Fiecare chunk în sine are o structură specifică, care ar trebui să fie afișată.


\section{Imaginea inițială a implementării la nivel înalt}

În dezvoltarea software de obicei are sens să soluționeze
problemele atunci când acestea trebuie fi abordate,
însă și procesul de gândire și modul în care programatorul ajunge la o soluție
tot are valoare dacă este partajat.
Lucrurile înregistrate aici nu sunt garantate să aducă la o soluție bună,
sau chiar la o soluție funcțională, iar anumite probleme legate de implementare
sunt cu totul ignorate, înainte de a fi de fapt întâmpinate în procesul de realizare a programului.

Așa este natura dezvoltării software.
Faptul este că planificarea și gândirea de prisos înainte de a începe a scrie codul
de obicei aduce la un design incorect sau inferior, deoarece în timpul programării
se descoper anumite probleme care nu au fost luate în considerare înainte,
din cauza că cunoașterea întregului sistem ca unul singur încă nu era atinsă de programator.
Din această cauză nu ar fi înțelept să se planifice întregul sistem
dinainte, și numai după aceasta să se înceapă procesul de dezvoltare.
Însă, planificarea la nivel înalt a structurii aplicației,
adică luarea în considerare a cerințelor și folosirea experienței și intuiției pentru
a putea ajunge la o reprezentare mentală la nivel înalt a aplicației,
este totuși utilă pentru a putea descompune problema în subprobleme mai mici,
și a putea începe să se implementeze o soluție.
 
Acestea fiind spuse, ar trebui să se evidențieze imaginea sistemului la nivel înalt a autorului,
la care el a ajuns pe baza intuiției derivate din experiență și pe baza cerințelor.

În primul rând, aplicația ar trebui să fie împărțită în mai multe module:
\begin{itemize}
  \item 
  \textbf{Modulul de acces la fișiere după poziții absolute} care
  să gestioneze și cache-area segmentelor vizualizate în momentul dat,
  precum și ștergerea segmentelor nedorite din memorie,
  și citirea datelor din fișier atunci când este nevoie.
  Acest modul este o abstracție importantă pentru a simplica interacțiunea cu
  fișierele PNG, ca acest proces să fie mai declarativ și ca module băzate pe
  acesta să nu fie nevoite a face operațiuni de nivel foarte jos.

  \item
  \textbf{Arborele sintactic al fișierului PNG}.
  Ideal, s-ar trebui să se păstreze în memorie doar arborele pentru acele chunk-uri
  care sunt vizualizate în momentul dat, și să se șteargă restul.
  Acest lucru ar putea fi implementat așa ca modulul de interfață cu utilizatorul să semnalizeze
  ce segmente din fișier acesta dorește să vizualizeze, indicate prin poziții absolute 
  de start și de sfârșit ale acestora în fișier.
  Modulul acesta ar permite accesul la niște informații structurate despre fișier.

  \item
  Propriu zis \textbf{parser}, care să poată transforma
  datele din fișier într-un arbore sintactic, la cerere.
  Acesta nu trebuie să existe ca un obiect, ajung și niște funcții care să poată
  fi apelate pentru a realiza această transformare.

  \item
  \textbf{Modulul de interfață cu utilizatorul}.
  Acesta ar trebui să folosească arborele sintactic al fișierului dat pentru
  a arăta utilizatului informațiile păstrate în arbore într-o formă captivantă.
  Acest modul ar trebui să aibă control asupra pozițiilor de start și de sfârșit
  ale segmentelor din fișier care sunt vizualizate în momentul dat, 
  permitând utilizatorului să actualizeze acestea în dependență 
  de care chunk-uri sunt afișate în interfață.
\end{itemize}

După cum s-a argumentat înainte, design-ul acesta este inițial,
și implicările de folosire a acestuia deliberat nu au fost analizate profund.


\chapterConclusionSection{implementation_chater_title}


\unnumberedChapter{Concluziile Finale și Recomandările}


\newpage
\markpage{usefulStuffEnd}


% Bibliography
\bibliographystyle{plain}
\bibliography{bibliography}
\addcontentsline{toc}{chapter}{\bibname}

% Appendices
\appendix

% Number with arabic numbers instead of Roman
\renewcommand{\thechapter}{\arabic{chapter}}
% Prepend Anexa to section names, center them
\titleformat{\section}[block]{\normalfont\normalsize\bfseries\filcenter}{Anexa \thesection~}{0pt}{}

% Every section on new page
% \newcommand{\sectionbreak}{\clearpage}

% Since we've got just a single chapter in the appedices,
% but which is also the name of the Appendix chapter, it should be omitted.
% Makes little sense, but ok I guess.
\setcounter{chapter}{1}

\unnumberedChapter{Anexe}
% insert appendices here

\end{document}