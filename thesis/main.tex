\documentclass[a4paper,12pt]{report}
\usepackage{config}

% Description
\newcommand{\authorName}{CURMANSCHII Anton}
\newcommand{\thesisTitle}{PNG}
\newcommand{\uniGroupName}{IA1901}
\newcommand{\thesisType}{master}
\newcommand{\programulDeStudii}{master}
\newcommand{\identificatorulCursului}{0613.5 Informatică aplicată}

% Hardcoded for now
\newcommand{\anexeCount}{20}
\newcommand{\conferencesList}{I don't know what to write here}

\begin{document}

% Prevent spills over margin
\sloppy

\begin{titlepage}
  \vspace*{\fill}
  \begin{center}
      \vspace*{1cm}

      \large
      \uppercase{\textbf{UNIVERSITATEA DE STAT DIN MOLDOVA\\}}

      \normalsize
      \uppercase{\textbf{FACULTATEA DE MATEMATICĂ și INFORMATICĂ}}
      \vspace{0.1cm}

      \normalsize
      \uppercase{\textbf{DEPARTAMENTUL INFORMATICĂ}}
      \vspace{3.0cm}

      \large
      \textbf{\expandafter{\authorName}}
      \vspace{1.5cm}

      \Large
      \textbf{\uppercase\expandafter{\thesisTitle}}
      \vspace{0.75cm}

      \large
      \textbf{\uppercase\expandafter{\identificatorulCursului}}
      \vspace{0.5cm}

      \normalsize
      \textbf{Teză de \programulDeStudii{}}
  
    \end{center}
  \vfill

  \normalsize
  
  \begin{flushleft}
    \begin{tabular}{p{4.3cm} p{4cm} p{8cm}}
      Director de departament: & \signatureField{4cm} & CAPCELEA Titu, \\
                               &                      & doctor în științe fizico-matematice, \\
                               &                      & conferențiar universitar \\
      Conducător științific:   & \signatureField{4cm} & \uppercase\expandafter{\conducatorNume\relax} \conducatorPrenume{}, \\
                               &                      & doctor, conferențiar universitar \\
      Autor:                   & \signatureField{4cm} & \authorName{}, \\
                               &                      & masterand din grupa \uniGroupName{} \\
    \end{tabular}
      
  \end{flushleft}

  \vspace{1cm}

  \begin{center}
      \textbf{\location{} -- \year{}}
  \end{center}

\end{titlepage}


\clearpage
\tableofcontents

\clearpage
\unnumberedChapter{Lista Abrevierelor}
\begin{acronym}
  \acro{CI/CD}{Continuous Integration / Continuous Delivery}
  \acro{GUI}{Graphical User Interface}
  \acro{PNG}{Portable Network Graphics}
  \acro{I/O}{Input/Output}
  \acro{MVP}{Produs Minim Viabil}

\end{acronym}


\clearpage
\unnumberedChapter{Adnotare}

\textbf{la teza de \thesisType ``\thesisTitle'', a studentului \authorName{}, grupa \uniGroupName{}, programul de studii \programulDeStudii.}

\textbf{Structura tezei.}
Teza constă din: Introducere, \total{num_chapters} capitole, Concluzii generale și recomandări, Bibliografie \bibliographyEntryCount{} titluri.
Textul de bază cuprinde \usefulPageCount{} de pagini și \anexeCount{} de anexe.

\textbf{Cuvinte-cheie:} PNG

\textbf{Actualitatea.}
PNG este un format de imagini folosit pe scară largă pe Internet,
în dezoltarea aplicațiilor grafice, în jocuri, și în multe alte domenii.
Înțelegerea structurii formatului poate fi utilă pentru a-l putea extinde, 
a-l utiliza într-un mod mai eficient și acesta poate servi drept referință pentru 
crearea unor formate noi.

\textbf{Scopul și obiectivele cercetării}.



\textbf{Rezultatele preconizate și obținute} rezumă în: (1) (2) (3)

\textbf{Problemele importante rezolvate} sunt:

\textbf{Valoarea aplicativă.}

Rezultatele obținute au fost raportate la Conferința (-ele) \textbf{\conferencesList}.

% I think this one's required to be capitalized.
\unnumberedChapter{INTRODUCERE}

\markpage{usefulStuffBegin}

\textbf{Actualitatea și importanța temei.}

\textbf{Scopul și obiectivele.}

\textbf{Suportul metodologic și tehnologic.}

\textbf{Noutatea stiințifică/originalitatea.}

\textbf{Valoarea aplicativă.}

\textbf{Sumarul tezei.}

Primul capitol, \nameref{intro_chapter_title}, aduce informații generale.
Al doilea capitol, \nameref{architecture_chapter_title}, concretizează cerințele.
Al treilea capitol, \nameref{implementation_chapter_title}, urmează implementarea.


\chapter{Capitol introducere}\label{intro_chapter_title}

\chapterConclusionSection{intro_chapter_title}


\chapter{Arhitectura Aplicației}\label{architecture_chapter_title}

\chapterConclusionSection{architecture_chapter_title}


\chapter{Implementarea Sistemului}\label{implementation_chapter_title}

În continuare, se va prezenta implementarea unui decoder PNG care va fi folosit pentru
a vizualiza formatul PNG printr-o aplicație \ac{GUI}.
Decoderul va fi implementat în limbajul de programare Zig\cite{zig}
după specificația \ac{PNG}, versiunea 1.2\cite{png_spec}.
Interfața grafică va fi implementată folosind Raylib\cite{raylib}, tot în Zig.

Pentru a facilita interacțiunea cu stream-uri de date și a crea o abstracție în acest sens,
a fost dezvoltată o mică bibliotecă.
Experiența anterioară cu biblioteca \texttt{System.IO.Pipelines}\cite{system_io_pipelines}
din C# a relevat mai multe avantaje în utilizarea unei astfel de abordări:
\begin{itemize}
  \item
  Implementarea corectă core să țină cont de toate cazurile-limită
  este facilitată de gestionarea complexității de către
  bibliotecă, cum ar fi alocarea și ștergerea buferelor,
  sau înfășurarea datelor pe mai multe segmente consecutive.

  \item
  Flexibilitatea codului crește, permițând implementarea unui automat finit
  și o rerulare din stări salvate anterior.

  \item
  Eficiența crescută prin soluționarea cazurilor de copiere a datelor în bufere temporare
  de o mărime neoptimă pentru cititor.

  \item
  Centralizarea logicii de citire într-un ciclu unificat,
  reducând duplicarea codului.

  \item
  Separarea completă a modulelor de citire și scriere,
  permițând procesarea datelor în fire separate,
  ceea ce elimină timpul pierdut la \ac{I/O} în timpul procesării.
\end{itemize}

Un prototip al unei biblioteci similare în Zig a fost dezvoltat,
omitând însă separarea modulelor de citire și scriere pentru a reduce complexitatea.
Această parte poate fi adăugată ulterior, nefiind esențială pentru un \ac{MVP}.

\section{Ideea aplicației la nivel înalt}

Având la bază o abordare explorativă, s-a decis programarea aplicației
pornind de la o imagine vagă a rezultatului dorit,
însă cu o direcție clară în ceea ce privește obiectivul general.

Având în vedere faptul că scopul principal al aplicației
este de a facilita explorarea structurii fișierelor PNG,
s-ar dori ca funcționalitatea să fie direct
influențată și determinată de această structură.
Designul interfeței grafice este conceput pentru a reflecta
caracteristicile specifice ale fișierelor PNG.

Fișierele PNG sunt compuse din secvențe denumite chunk-uri,
fiecare având o funcție specifică.
De exemplu, chunk-ul \texttt{IHDR} conține informații vitale despre imagine,
precum lățimea, înălțimea, numărul de biți per pixel ș.a.
Se intenționează ca interfața aplicației să pună
în evidență aceste chunk-uri, oferind detalii despre ele utilizatorilor
și permițând modificarea valorilor într-un mod
care nu afectează integritatea fișierului.
Informațiile ar fi prezentate per chunk, atunci când utilizatorul îl alege,
iar fiecare chunk trebuie să fie evidențiat, pentru a-l putea alege și vizualiza.

Pentru a sublinia aspectul de explorare a \textit{formatului} PNG,
și nu doar manipularea imaginilor în acest format,
se va include în aplicație un editor hexazecimal.
Acesta va prezenta valorile fiecărui octet și
va oferi reprezentări vizuale pentru a facilita distingerea diferitelor chunk-uri.
O caracteristică utilă a editorului va fi funcționalitatea
de a minimiza anumite chunk-uri, utilizând o iconiță dedicată,
deoarece multe dintre ele conțin date de pixeli
care nu sunt esențiale pentru înțelegerea formatului.

La selectarea unui chunk, vor fi afișate informații detaliate despre acesta,
incluzând numele, descrierea și lungimea.
Este crucial ca fiecare chunk să fie prezentat cu o structură specifică,
pentru a oferi o înțelegere completă a compoziției și funcționalității fișierelor PNG.


\section{Imaginea inițială a implementării la nivel înalt}

În dezvoltarea software, este adesea practic să se abordeze problemele pe măsură ce apar.
Însă procesul de gândire și modul în care se ajunge la soluții
pot adăuga valoare dacă sunt împărtășite.
Deși nu există garanția că ideile înregistrate vor duce
la o soluție optimă sau chiar una funcțională,
și anumite probleme de implementare sunt ignorate la această etapă inițială,
această abordare reflectă natura evolutivă a dezvoltării software.

Excesul de planificare și gândire înainte de începerea scrierii codului
poate adesea conduce la un design incorect sau inferior.
Acest lucru se datorează faptului că în timpul programării pot apărea probleme neanticipate,
din cauza lipsei unei viziuni complete asupra întregului sistem.
Din acest motiv, planificarea detaliată a întregului sistem înainte de începerea dezvoltării
nu este întotdeauna cea mai înțeleaptă abordare.
Cu toate acestea, planificarea la nivel înalt a structurii aplicației,
ținând cont de cerințe și pe baza experienței,
este utilă pentru a descompune problema în subprobleme gestionabile
și pentru a începe implementarea unei soluții.

Pentru a reflecta această viziune, s-a stabilit ca aplicația
să fie împărțită în mai multe module esențiale:
\begin{itemize}
  \item 
  \textbf{Modulul de acces la fișiere după poziții absolute},
  care va gestiona cache-ul segmentelor vizualizate în prezent,
  ștergerea segmentelor nedorite din memorie și citirea datelor din fișier la necesitate.
  Acest modul este o abstracție cheie pentru a simplifica interacțiunea cu fișierele PNG.

  \item
  \textbf{Arborele sintactic al fișierului PNG},
  care ar trebui să păstreze în memorie doar informațiile despre
  chunk-urile vizualizate la un moment dat.
  Acesta ar permite accesul la informații structurate despre fișier
  și ar fi interfațat cu modulul de interacțiune cu utilizatorul
  pentru a indica ce segmente din fișier sunt vizualizate.

  \item
  \textbf{Parser-ul}, responsabil pentru transformarea datelor
  din fișier într-un arbore sintactic la cerere.
  Acest modul poate consta dintr-un set de funcții specifice pentru această transformare.

  \item
  \textbf{Modulul de interfață cu utilizatorul},
  care va folosi arborele sintactic al fișierului pentru
  a prezenta utilizatorului informații într-o formă atractivă și interactivă.
  Modulul va avea control asupra segmentelor din fișier care sunt vizualizate,
  permitând utilizatorului să actualizeze aceste segmente
  în funcție de chunk-urile afișate în interfață.
\end{itemize}

Așa cum s-a discutat, acest design inițial este un punct de plecare,
iar implicațiile utilizării acestuia urmează să fie analizate pe parcursul dezvoltării.


\chapterConclusionSection{implementation_chater_title}


\unnumberedChapter{Concluziile Finale și Recomandările}


\newpage
\markpage{usefulStuffEnd}


% Bibliography
\bibliographystyle{plain}
\bibliography{bibliography}
\addcontentsline{toc}{chapter}{\bibname}

% Appendices
\appendix

% Number with arabic numbers instead of Roman
\renewcommand{\thechapter}{\arabic{chapter}}
% Prepend Anexa to section names, center them
\titleformat{\section}[block]{\normalfont\normalsize\bfseries\filcenter}{Anexa \thesection~}{0pt}{}

% Every section on new page
% \newcommand{\sectionbreak}{\clearpage}

% Since we've got just a single chapter in the appedices,
% but which is also the name of the Appendix chapter, it should be omitted.
% Makes little sense, but ok I guess.
\setcounter{chapter}{1}

\unnumberedChapter{Anexe}
% insert appendices here

\end{document}